%Chapter for installing and setting up dabo

\chapter{Installation of the Dabo Development Environment}

This chapter details the process for getting your computer set up for developing 
applications using Dabo. While these instructions would work for your deployment 
targets as well, there are better ways to deploy applications than by using the 
instructions here. Dabo has dependencies on a number of external libraries, and 
while developing your applications you'll want to keep all those libraries - and Dabo 
- as current as possible. For deployment, you want better control over the versions 
in use. Deploying applications is covered in (need link).

\section{Downloading and Installing}

Dabo has dependencies on a number of external libraries, and those dependencies 
will vary depending on your choice of database and user-interface library. In general, 
you will want to install, in this order:
\begin{description}
	\item[Python] Python is available for download from http://www.python.org.  
	Please install the most recent stable version available. Follow the instructions 
	there for installation instructions for your platform. Python is certainly already 
	installed on your Linux or Macintosh system, but for Windows you may find that 
	you need to install it yourself. No matter what, please check to see if the version 
	you have is relatively recent and if not you should download and install the most 
	recent stable Python release.

	\item[SQLite] SQLite comes with Python 2.5 and higher. If you happen to be 
	running an earlier version of Python, then you'll need to install SQLite manually. 
	Do this by downloading pysqlite2 and running 'python setup.py install'.

	\item[wxPython] This is the standard user-interface toolkit for Dabo, and at the 
	time of this writing is required for building applications that present an interface 
	to the user. In the future, Dabo will support other user-interface toolkits as well, 
	but for now the only supported toolkit is wxPython. wxPython is in a state of 
	rapid development, so it is best to stay as current as you can with it. Download 
	and install the most recent stable version from http://www.wxpython.org.

	\item[Dabo] Get the most recent version of Dabo from http://dabodev.com. Be 
	sure to get the main Dabo package as well as dabodemo and daboIDE. Like most 
	other Python packages, Dabo uses distutils so a simple 'python setup.py install' 
	should get Dabo into your Python installation's site-packages directory, which is 
	where all third-party libraries for Python are normally installed.
\end{description}

\section{Testing Your Installation}

Now that you've downloaded and installed all the prerequisites, you need to run 
some tests to be reasonably sure everything is installed correctly. The tests involve 
interacting with your operating system's command line, which as a developer you 
really should try to get familiar with.

Microsoft Windows: Go to Start|Run and type 'cmd' <enter>.
Apple Macintosh: Navigate to your Applictions/Utilities directory and double-click on 
the Terminal application.
Linux/UNIX: Different distributions put this in different places. Look for xterm, terminal, 
or command-line in your desktop menu system.
	
Open up your command line, and type 'python'. You should get output like:	%NEED TO GET SETUP FOR STANDARD CODE OUTPUT SETUP
\begin{verbatim}
[pmcnett@sol book]$ python
Python 2.3.2 (#1, Oct  6 2003, 10:07:16)
[GCC 3.2.2 20030222 (Red Hat Linux 3.2.2-5)] on linux2
Type "help", "copyright", "credits" or "license" for more information.
>>>
You are now inside Python's command interpreter. Test to make sure that MySQLdb,
wxPython, and Dabo load correctly. If there are no errors, they are installed correctly.

>>> import wx
>>> import dabo
Dabo Info Log: Thu Sep  9 19:16:23 2004: No default UI set. (DABO_DEFAULT_UI)
>>>
\end{verbatim}
The message from Dabo is normal, and no errors happened during the import of the 
other packages, so everything is set up correctly on my system. 
