%History of Dabo Chapter

\chapter{History of Dabo}

Dabo is the result of a few years of research, starting in 2001 when I started taking 
an active interest in the Linux operating system and open source software in general. 
I had been using Microsoft Visual FoxPro to develop data-aware business applications 
for my clients, and with mixed messages coming from Microsoft and the FoxPro 
community as to the long-term viability of FoxPro as a product, I started looking for 
alternatives from the open source community, alternatives that would permit the 
development of powerful database applications for multi-platform deployment.

My quest led me first to Borland Delphi, which had just recently announced a 
prerelease version of Kylix, the Linux version of Delphi. This product would allow 
deployment of a semi-common codebase to one flavor of Linux (RedHat 7.x) and 
Windows. It had good data-aware controls, but only in the Enterprise version. The 
execution performance was pitiful, and code wasn't truly portable between platforms. 
The lack of a commitment from Borland to support Macintosh was the straw that 
made me look elsewhere.

The next stop was to take a serious look at Java, which does have a clean and 
elegant language and does run pretty much equally on all platforms, thanks to the 
Java Virtual Machine. I developed some prototype applications using the Swing 
components which performed equally horribly on all three platforms. But, it was easy 
development and deployment, although database integration was pretty obtuse 
compared to what I was used to in Visual FoxPro. My overall feeling was that Java 
may really take over the world, but it has a long way to go performance-wise, and 
it really isn't fun to code.

Somewhere about this time, I found out that Kylix was using a toolkit called Qt to 
provide the user interface components, and that Kylix was using the last-generation 
(Qt 2.x) instead of the newer, much nicer Qt 3.x version. I downloaded the GPL'd 
version of Qt for Linux, and followed some tutorials, and was able to build some very i
mpressive C++ applications that performed very well. I never got my head around 
C++, however, so I felt like I'd be painting myself into a corner by pursuing this 
angle. Also, database support required the Enterprise version of Qt, which is 
something like \$1200 per platform. Hardly free and open source.

Eventually, I came to know of a programming language called Python, which Ed 
Leafe had been using to power his website using a product called Zope. I found 
Python to be very intuitive to learn, with an easy readable syntax not unlike FoxPro. 
Python, on top of being free and open source, also comes with "batteries included", 
meaning that most everything you'd want to do comes in the standard library, 
including building user interfaces. Python also comes with three native data types 
that have to do with sequences - in other words, Python can represent database 
tables and fields natively. Come to find out, Python also provides an API for 
connecting to all kinds of database servers. In other words, Python comes with a 
lot of the pieces I'd been searching for over the past few years.

While it was great finding that Python had a lot of the pieces to my puzzle, it was 
another thing entirely realizing that putting all the pieces together into a workable 
whole would prove to be anything but easy. Yes, Python can connect to any 
database and retrieve and update data. Yes, Python has a graphical user interface. 
No, connecting to a given database isn't the same as connecting to a different 
database. And no, the user interface that comes with Python is not very modern.

In the Fall of 2003, I set out to create a framework for developing data-aware 
applications in Python, using a GUI toolkit I'd just learned about called wxWindows. 
I'd actually heard about wxWindows before, but had discounted it thinking that it 
was only for the Windows platform. They've since changed their name to wxWidgets 
at Microsoft's request, which may keep future developers from being as confused 
as I was. I named this framework 'Dabo', because it sounded fun and reminds me of 
words like 'data', 'business', 'application', and 'objects'. Also, we were watching 
Star Trek: Deep Space Nine at the time and I liked the Dabo Girls.

I ended up learning the wxWidgets toolkit pretty well, but created a pile of spaghetti 
code that had no separation between the database, the business rules, and the 
user interface. It was really a mess and completely unmaintainable. I set the project 
down for a few months, until I was contacted in March of 2004 by Ed Leafe, a 
long-time FoxPro guru that was looking for ways to move his skillset over to open 
source, multiplatform development - he was looking for the same things I'd been 
seeking over the previous couple years. I decided to share my code with him, along 
with a sample application, and he very diplomatically explained all the design 
problems with my approach. We came up with an agreement to redesign Dabo from 
the ground up, with a 3-tier model.

By May of 2004 we announced our work to the public, got a website and mailing 
lists, and encouragement from diverse areas of the open source, Python, and FoxPro 
communities. As I write this in September 2004, Dabo is under active development, 
the user interface is 80\% abstracted, 3 databases are supported, and a 
user-interface graphical designer is underway. It is already possible to create 
powerful data-aware applications and to deploy them to Windows, Linux, and 
Macintosh.
